\documentclass[UTF8]{ctexart}

\usepackage{ulem}
\usepackage{geometry}
\usepackage{amsmath}
\usepackage{multicol}

\title{梁的地理笔记}
\author{LWJ}
\date{转录开始于2022.5.28}
\geometry{a4paper,right=2.0cm,left=2.0cm,top = 2.0cm, bottom = 2.0cm}

\begin{document}

\maketitle

这篇地理笔记来自我的同学,梁。笔记主要记录了部分简答题的答题要点和\sout{奇技淫巧},以助于快速得分,建议在考前阅读。本篇笔记来源于上课知识点的整理,并不能涵盖所有简答题的考察范围,不过存在一些特殊问题的解答,可以与后续上传的新东方的简答题答题模板互补进行阅读与学习。

\par

\hfill Cicada000

\thispagestyle{empty}

\newpage

\setcounter{page}{1}

\begin{multicols}{2}

    \[
        \textbf{土壤肥力}
        \begin{cases}
            \text{厚度:地形平坦}\\
            \text{有机质}
            \begin{cases}
                \text{归还多(植被茂盛)}\\
                \text{分解慢(温度低、含水量高)}
            \end{cases}\\
            \text{矿物养分:成土母质(火山灰\dots)}
        \end{cases}
    \]

    \[
        \textbf{风力}
        \begin{cases}
            \text{水平气压梯度力}\\
            \text{地面摩擦力(植被覆盖)}\\
            \text{狭管效应}\\
            \text{风源地距离}
        \end{cases}
    \]

\end{multicols}

\par

\begin{multicols}{2}
    
    \[
        \textbf{湿地、沼泽成因}
        \begin{cases}
            \text{水汇入}
            \begin{cases}
                \text{河流汇入}\\
                \text{降水多}
            \end{cases}\\
            \text{水排不走}
            \begin{cases}
                \text{排出径流少}\\
                \text{蒸发少}\\
                \text{下渗不多}
            \end{cases}
        \end{cases}
    \]

    \[
        \textbf{渔场}
        \begin{cases}
            \text{洋流(寒暖流交汇)}\\
            \text{大陆架广阔(浅海光照足,光合作用强,含氧量大)}\\
            \text{入海径流(泥沙与营养物质)}\\
            \text{水温季节变化大(表层深层海水交换,营养物质上泛)}
        \end{cases}
    \]

\end{multicols}

\par 

\begin{multicols}{2}

    \[
        \textbf{春城}
        \begin{cases}
            \text{低纬度高海拔}\\
            \text{低纬度寒流流经}
        \end{cases}
    \]

    \[
        \textbf{灾情分析}
        \begin{cases}
            \text{基础设施}\\
            \text{灾害强度}\\
            \text{发生时间}\\
            \text{人口密度、经济密度}\\
            \text{防灾意识、救灾能力}
        \end{cases}
    \]

    \[
        \textbf{河流水文特征}
        \begin{cases}
            \text{流量及变化}
            \begin{cases}
                \text{流量:气候、流域面积、支流}\\
                \text{变化:补给类型}
            \end{cases}\\
            \text{流速:地形}\\
            \text{含沙量:植被覆盖率 \dots (见P3)}\\
            \text{结冰期:温度}
        \end{cases}
    \]
\end{multicols}

\par

\begin{multicols}{2}

    \[
        \textbf{火山}
        \begin{cases}
            \text{危害}
            \begin{cases}
                \text{影响气候,火山灰削弱太阳辐射}\\
                \text{污染大气}\\
                \text{破坏生态环境}\\
                \text{造成生命财产损失,影响人体健康}
            \end{cases}\\
            \text{有利}
            \begin{cases}
                \text{农业:肥沃土壤}\\
                \text{旅游业:火山景观与温泉}\\
                \text{采矿业:硫磺与有色金属、岩浆岩}\\
                \text{制造陆地,增大岛屿面积}\\
                \text{缓解气候变暖}
            \end{cases}
        \end{cases}
    \]

    \[
        \textbf{滑坡泥石流成因}
        \begin{cases}
            \text{地形:坡度大}\\
            \text{地质:岩层破碎}\\
            \text{气候:暴雨}\\
            \text{植被:稀疏}\\
            \text{人类活动:矿渣、固体废弃物堆积}
        \end{cases}
    \]

\end{multicols}

\begin{multicols}{2}
    
    \[
        \textbf{种植园农业}
        \begin{cases}
            \text{水热条件优越}\\
            \text{劳动力丰富廉价}\\
            \text{消费市场广阔}\\
            \text{海运便利}
        \end{cases}
    \]

    \[
        \textbf{洪涝成因}
        \begin{cases}
            \text{来水}
            \begin{cases}
                \text{降水集中,多暴雨}\\
                \text{多支流}\\
                \text{流域面积大}\\
                \text{植被稀疏}
            \end{cases}\\
            \text{去水}
            \begin{cases}
                \text{河道弯曲}\\
                \text{河道淤塞抬高}\\
                \text{地势低洼,排水不畅}
            \end{cases}
        \end{cases}
    \]

\end{multicols}

\newpage

\begin{multicols}{2}

    \[
        \textbf{高新技术产业区位}
        \begin{cases}
            \text{人才(高等院校与科研机构)}\\
            \text{交通(利于产品运输)}\\
            \text{环境(吸引人才)}\\
            \text{协作(产业基础完善,通讯发达)}
        \end{cases}
    \]

    \[
        \textbf{港口}
        \begin{cases}
            \text{水域}
            \begin{cases}
                \text{航行(水深)}\\
                \text{停泊(风浪小,海岸浅)}
            \end{cases}\\
            \text{陆域}
            \begin{cases}
                \text{筑港(沿岸平坦)}\\
                \text{经济腹地(经济发展程度)}\\
                \text{依托城市}
            \end{cases}
        \end{cases}
    \]

\end{multicols}

\par

\begin{multicols}{2}
    
    \[
        \textbf{三角洲}
        \begin{cases}
            \text{泥沙}
            \begin{cases}
                \text{植被覆盖率低}\\
                \text{土质疏松}\\
                \text{降水多且集中}\\
                \text{流速快,侵蚀}\\
                \text{流量大}
            \end{cases}\\
            \text{沉积}
            \begin{cases}
                \text{流速慢,沉积}\\
                \text{海水,海风顶托作用}
            \end{cases}\\
            \text{海浪侵蚀}
        \end{cases}
    \]

    \[
        \textbf{湖泊水文特征}
        \begin{cases}
            \text{水量与水位变化}\\
            \text{含沙量}\\
            \text{盐度}\\
            \text{结冰状况}\\
            \text{湖泊流(大湖泊才有)}
        \end{cases}
    \]

    \[
        \textbf{乳畜业区位}
        \begin{cases}
            \text{饲料}\\
            \text{市场}
        \end{cases}
    \]

\end{multicols}

\par

\begin{multicols}{2}
    
    \[
        \textbf{大坝选址}
        \begin{cases}
            \text{水能}
            \begin{cases}
                \text{降水量}\\
                \text{落差}\\
            \end{cases}\\
            \text{坝址}
            \begin{cases}
                \text{峡谷地形(工程量小,造价低)}\\
                \text{地质稳定(避断层、喀斯特地貌)}\\
                \text{适合梯级开发}
            \end{cases}\\
            \text{社会经济:迁人口、技术、资金、市场}
        \end{cases}
    \]

    \[
        \textbf{老工业基地问题}
        \begin{cases}
            \text{经济体制落后,产业结构单一}\\
            \text{生产设备老化,生产工艺落后}\\
            \text{矿产资源枯竭}\\
            \text{环境污染严重}\\
            \text{水资源短缺}
        \end{cases}
    \]

\end{multicols}

\par

\begin{multicols}{2}
    
    \[
        \textbf{老工业基地振兴}
        \begin{cases}
            \text{优化产业结构,建立现代产业体系}\\
            \text{推动技术发展,提升自主创新能力}\\
            \text{加强基建,完善交通网络}\\
            \text{推进转型,促进可持续发展}\\
            \text{保护生态环境,发展绿色经济}
        \end{cases}
    \]

    \[
        \textbf{生产地区专门化}\\
        \textbf{优势}
        \begin{cases}
            \text{充分发挥地区资源,挖掘增产潜力}\\
            \text{利于集中农业机械与技术装备}\\
            \text{利于推广产业技术}\\
            \text{利于提高土地利用率}\\
            \text{提高农业生产效率与商品率}
        \end{cases}
    \]

\end{multicols}

\par

\begin{multicols}{2}
    
    \[
        \textbf{盐场布局}
        \begin{cases}
            \text{平坦开阔海滩(淤泥质海滩最佳)}\\
            \text{蒸发旺盛,日照充足,风力大}
        \end{cases}
    \]

    \[
        \textbf{大气污染}
        \begin{cases}
            \text{排放多}\\
            \text{难扩散}
            \begin{cases}
                \text{地区封闭}\\
                \text{风力小}\\
                \text{气压高}\\
                \text{逆温作用}
            \end{cases}
        \end{cases}
    \]

\end{multicols}

\par

\begin{multicols}{2}

    \[
        \textbf{光化学烟雾}
        \begin{cases}
            \text{光照强}\\
            \text{风力小}\\
            \text{逆温作用}
        \end{cases}
    \]

    \[
        \textbf{选择运输方式}
        \begin{cases}
            \text{运输方式优点}\\
            \text{货物特点}\\
            \text{可行性}
        \end{cases}
    \]
    
\end{multicols}

\par

\begin{multicols}{2}
    
    \[
        \textbf{地下水位下降}
        \begin{cases}
            \text{地面沉降,形成漏斗区,影响建筑}\\
            \text{沿海海水倒灌,影响水质,土地盐碱化}\\
            \text{地形低洼影响排洪}\\
            \text{受风暴潮影响更大}
        \end{cases}
    \]

    \[
        \textbf{地貌形成因子}
        \begin{cases}
            \text{内力作用}
            \begin{cases}
                \text{板块构造}\\
                \text{地质构造}
            \end{cases}\\
            \text{外力作用}
            \begin{cases}
                \text{流水作用}\\
                \text{冰川作用}\\
                \text{风力作用}
            \end{cases}
        \end{cases}
    \]

\end{multicols}

\par

\begin{multicols}{2}

    \[
        \textbf{河深增加}
        \begin{cases}
            \text{补给水位上升}\\
            \text{下切侵蚀,河床加深}
        \end{cases}
    \]

    \[
        \textbf{黑风暴}
        \begin{cases}
            \text{成因}
            \begin{cases}
                \text{过度开垦放牧,植被破坏}\\
                \text{长期干旱}\\
                \text{风力大}
            \end{cases}\\
            \text{对策:限耕、休耕、免耕、轮作}
        \end{cases}
    \]

\end{multicols}

\par

\begin{multicols}{2}
    
    \[
        \textbf{农业区位}
        \begin{cases}
            \text{交通}\\
            \text{农业生产条件(水、热、土)}\\
            \text{农业生产技术}\\
            \text{市场需求}\\
            \text{冷藏存储技术}\\
            \text{城市发展}
        \end{cases}
    \]

    \[
        \textbf{河流含沙量}
        \begin{cases}
            \text{上游流域面积(影响泥沙来源)}\\
            \text{上游地区坡度(侵蚀与携带能力)}\\
            \text{降水强度}\\
            \text{植被覆盖率}\\
            \text{土壤质地,河床性质}\\
            \text{经过沼泽湖泊等泥沙沉积}
        \end{cases}
    \]

\end{multicols}

\text{最后编辑于\today}

\end{document}
